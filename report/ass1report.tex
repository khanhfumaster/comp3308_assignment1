\documentclass{article}
\usepackage{fancyhdr} 
\usepackage{lastpage}
\usepackage{mathtools}
\usepackage{extramarks}
\usepackage{graphicx}
\usepackage{listings}
\usepackage{courier}
\usepackage{lipsum} 
\usepackage{enumerate}
\usepackage{amsmath}
\usepackage{url}

% Margins
\topmargin=-0.45in
\evensidemargin=0in
\oddsidemargin=0in
\textwidth=6.5in
\textheight=9.0in
\headsep=0.25in
\linespread{1.1} % Line spacing

% Set up the header and footer
\pagestyle{fancy}
\lhead{COMP3308}
\chead{Introduction to Artificial Intelligence}
\rhead{Assignment 1}
\lfoot{}
\cfoot{\thepage}
\rfoot{Woo Hyun Jung 310250811 \\  Khanh Cao Quoc Nguyen 311253865} 
\renewcommand\headrulewidth{0.4pt}
\renewcommand\footrulewidth{0.4pt}
\renewcommand{\tt}{\texttt}
\setlength\parindent{0pt} 

\title{COMP3308 Assignment 1 \\ Predicting Diabetes}
\author{Woo Hyun Jung 310250811 \\  Khanh Cao Quoc Nguyen 311253865}
\date{}
\begin{document}
\maketitle
\thispagestyle{empty}
\newpage

\section{Aim}
The aim of our study is to predict whether a new patient will test positive for diabetes (class 1). The study is important because it can help to determine what sort of factors can lead to having diabetes, and as a result, what can be adjusted in a patient's lifestyle to lessen the chance of diabetes.

\section{Data}
\subsection{Dataset}
The data set we used was the Pima Indians Diabetes Database found at \url{http://archive.ics.uci.edu/ml/machine-learning-databases/pima-indians-diabetes/}. \\
There are nine attributes: 
\begin{enumerate}[1.]
\item Number of times pregnant
\item Plasma glucose concentration a 2 hours in an oral glucose tolerance test
\item Diastolic blood pressure (mm Hg)
\item Triceps skin fold thickness (mm)
\item 2-Hour serum insulin (mu U/ml)
\item Body mass index (weight in kg/(height in m)$^2)$
\item Diabetes pedigree function
\item Age (years)
\item Class variable (0 or 1)
\end{enumerate} 
and two classes:
\begin{enumerate}[1.]
\item class1 (testing positive for diabetes)
\item class0 (testing negative for diabetes)
\end{enumerate} 

\subsection{Data preparation}
\subsubsection{Manual preprocessing}
The raw \tt{.data} file was preprocessed manually in multiple ways. Firstly we had to come up with simple names for the nine attributes:
\begin{enumerate}[1.]
\item num\_pregnant
\item plasma\_glucose\_concentration
\item diastolic\_blood\_pressure
\item tricep\_skin\_fold\_thickness
\item 2h\_serum\_insulin
\item bmi
\item diabetes\_pedigree\_function
\item age
\item class
\end{enumerate}
Using these names, we added a header row the \tt{.data} file. We then changed the values of the class column to class0 and class1 instead of 0 and 1. This file was then saved a \tt{.csv} file for later convenience.

\subsubsection{Missing Values}
We wrote a script \tt{missing\_values.py} that dealt with the missing values in the following attributes:
\begin{enumerate}[1.]
\item plasma\_glucose\_concentration
\item diastolic\_blood\_pressure
\item tricep\_skin\_fold\_thickness
\item 2h\_serum\_insulin
\item bmi
\item diabetes\_pedigree\_function
\end{enumerate} 

We determined that these attributes consisted of missing values because 0 is an impossible or unrealistic value.\\

This script outputs two \tt{.csv} files, one with missing values taking the calculated average of its respective attribute(\tt{pima-indians-diabetes-avg.csv}) and the other with any instances with missing values omitted(\tt{pima-indians-diabetes-omit.csv}).\\

We noticed that if we omit the instances with missing values we would be left with 392 instances out of the original 768 (only 51\%!), so we decided to use the averaged output file for the rest of the assignment.

\subsubsection{Normalisation}
Once this new file was loaded into Weka, each of the attributes were normalised in the range $[0, 1]$. The final output file was saved as \tt{pima.csv}

\subsection{Attribute selection}
With the \tt{pima.csv} loaded into Weka, we used the \tt{CfsSubsetEval} attribute evaluator with the \tt{BestFirst} search method available on the \tt{Select attributes} tab to determine the best subset of features based on how good the individual feature are at predicting the class and how much they correlate with the other features.
The selected attributes were:
\begin{enumerate}[1.]
\item plasma\_glucose\_concentration
\item 2h\_serum\_insulin
\item bmi
\item diabetes\_pedigree\_function
\item age
\end{enumerate} 

We then proceed to use Weka's preprocessing features to remove the attributes that were not selected and saved the file as \tt{pima-CFS.csv}.

\section{Results and Discussion}

\subsection{Results from Weka}
\begin{table}[h]
\begin{tabular}{llllllll}
\hline
 									& ZeroR    & 1R         & 1-NN     & 5-NN     & NB         & DT        & MLP \\ \hline
No feature selection 					& 65.1042 & 70.8333 & 67.7083 & 74.7396 & 75.0000 & 72.2656 & 75.3906 \\
Correlation-based feature selection 	& 65.1042 & 70.8333 & 69.0104 & 74.6094 & 76.4323 & 73.9583 & 75.7813 \\ \hline
\end{tabular}
\caption{Results from Weka (all values in \%)}
\end{table}

ZeroR predicts class0 with the same percentage in both sets of results as it just counts which instance is more popular. In our datasets, we had 500 instances of class0 and 268 of class1, giving us the same result of 65.1042\%. Similarly, 1R picked the plasma\_glucose\_concentration attribute and counted instances, giving us the same result of 70.8333\%. \\
For all the other algorithms, the performance was generally better with the feature selection. The 5-NN was much more accurate than the 1-NN in both sets, as would be expected by the selection and comparison of more neighbours. NB and MLP were the most accurate algorithms, possibly due to their effectiveness on such simple datasets.  \\
MLP displayed a training time of 0.47s on the complete data (0.24s on CFS data), while all other algorithms ran instantly (displayed 0s), due to the training and backpropagation algorithms. We believe that these low running times are due to the small size of the datasets along with the relative simplicity of the algorithms.  \\

\subsection{Results from own implementation}
\begin{table}[h]
\centering
\begin{tabular}{@{}llll@{}}
\hline
 									& My1-NN & My5-NN & MyNB \\ \hline
No feature selection 					& 68.7645 & 75.2546 & 75.3828 \\
Correlation-based feature selection 	& 68.4979 & 76.0509 & 76.6883 \\ \hline
\end{tabular}
\caption {Results of implementations of kNN and NB (all values in \%)}
\end{table}

Once again, feature selection generally produced more accurate results from our algorithms. Our implementations had very similar results to the Weka output, which was pleasant to see. 

\subsection{General discussion}
Weka's CFS provided us with a set of five attributes as seen in section 2.3 above. The selection was somewhat intuitive; the excluded attributes were num\_pregnant, diastolic\_blood\_pressure and tricep\_skin\_fold\_thickness. We could not reason why the number of pregnancies or minimum blood pressure would contribute to or result from diabetes. Tricep skin fold thickness, however, is a measure of body fat percentage, which is related to BMI (a feature that was selected) and may have been excluded due to its similarity and possibly redundant nature in the dataset.
In both sets of results, it was apparent that correlation-based feature selection was beneficial - using the CFS results in both Weka and our own algorithms provided more accurate results.  

\section{Conclusions}
Firstly, it should be noted that all the attributes tested were relevant in predicting diabetes (although some may have not been intuitively so for us). This is due to the fact that the CFS results were only slightly more accurate than the results with no feature selection (up to around 1.8\% in both result sets). However, the use of CFS was beneficial as it allowed us to narrow down the data sets to the most relevant attributes and get a better performance.  \\

For future experiments and research in this topic, it would be ideal to have less fragmented data and missing values. This will allow us to create a more accurate and realistic classifier where it does not need to take an average value of an attribute. 

\section{Reflection}
The most important thing we learned from this assignment was the aspect of teamwork. We split the tasks we had to do and took turns doing various parts of coding or writing. It was especially beneficial when we could have each other double check code or results which led to us both having a more correct understanding of the implementation or results at hand. Also, splitting the work allowed us to have a more fun and easier experience with the assignment. \\

Another thing we learnt was that artificial intelligence can be very interesting and quite fun to code. It has given us ideas to build other classifiers for a variety of different research topics. It has ultimately given us a tool for analysis and prediction for future research and development.

\section{Instructions}
Our implementation of K Nearest Neighbour, Naive Bayes and 10\-fold stratified cross validation was written in Python 2.7.5.\\

The \tt{ai\_main.py} file runs 10-fold stratified cross validation on K Nearest Neighbour ($k=1$ and $k=5$) and Naive Bayes on a given \tt{.csv} file.\\

To run our implementation, you must navigate to the directory with the files and ensure that \tt{pima.csv} and \tt{pima-CFS.csv} are present in the directory. Then run:
\begin{lstlisting}
python ai_main.py pima.csv
python ai_main.py pima-CFS.csv
\end{lstlisting}

This should output \tt{pima-folds.csv} and \tt{pima-CFS-folds.csv} respectively, which are the stratified folds.












\end{document}